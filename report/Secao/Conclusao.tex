\section{Conclusão}
    \label{sec:conc}
O algoritmo genético implementado foi capaz de obter as rotas com menor distância dentro do problema proposto de uma rota de viajem circular com 10 cidades, com ponto de partida fixo. A distância mínima de 128 foi atingida para duas rotas: \textbf{Brasília -> Caracas -> Bogotá -> Lima -> Santiago -> Salvador -> Porto Alegre -> São Paulo -> Rio de Janeiro -> Belo Horizonte -> Brasília}, e \textbf{Brasília -> Belo Horizonte -> Rio de Janeiro -> São Paulo -> Porto Alegre -> Salvador -> Santiago -> Lima -> Bogotá -> Caracas -> Brasília}.

As estratégias de reprodução como a combinação e a mutação devem ser pensadas e escolhidas com a finalidade de otimizar os descendentes que irão substituir em parte a população anterior para dar origem a uma nova geração. No algoritmo proposto, a evolução das gerações promovem uma menor média da distância a percorrer (Figuras~\ref{fig:exp_best} e~\ref{fig:exp_worst}), o que gera melhor ajuste dos resultados e distâncias mínimas cada vez menores.

A importância em analisar o comportamento do algoritmo ante as constantes iniciais como, por exemplo, tamanho da população e probabilidades de combinação e mutação foi evidenciada na Seção~\ref{sec:exp}. A escolha dos parâmetros que melhor se ajustam à problemática de um algoritmo genético permite a obtenção de resultados desejados ou aproximados dentro do número de gerações limite.

Portanto, os algoritmos genéticos são uma importante abordagem para o tratamento de problemas onde a otimização está presente. Por meio de ajustes através das gerações, os resultados obtidos tendem a evoluir em busca de melhor adaptação ao contexto imposto por um dado problema.